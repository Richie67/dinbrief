%
% \iffalse meta-comment
% $Id: dbaddon.dtx 1.8 1996/10/29 20:55:46 Gussmann Exp $
%
% $Log: dbaddon.dtx $
% Revision 1.8  1996/10/29 20:55:46  Gussmann
% - Some comments added.
%
% Revision 1.7  1996/01/09 20:10:38  Gussmann
% - BITNET emails removed.
%
% Revision 1.6  1995/10/05 23:35:59  Gussmann
% *** empty log message ***
%
% Revision 1.5  1995/08/15 23:49:53  Gussmann
% Upload message changed.
%
% Revision 1.4  1995/08/15 23:48:12  Gussmann
% Upload message added.
%
% Revision 1.3  1995/07/06 02:43:41  Gussmann
% Umlauts changed.
%
% Revision 1.2  1995/07/06 02:40:55  Gussmann
% *** empty log message ***
%
% \fi
%
% \CheckSum{4752}
%%
%% \CharacterTable
%%  {Upper-case    \A\B\C\D\E\F\G\H\I\J\K\L\M\N\O\P\Q\R\S\T\U\V\W\X\Y\Z
%%   Lower-case    \a\b\c\d\e\f\g\h\i\j\k\l\m\n\o\p\q\r\s\t\u\v\w\x\y\z
%%   Digits        \0\1\2\3\4\5\6\7\8\9
%%   Exclamation   \!     Double quote  \"     Hash (number) \#
%%   Dollar        \$     Percent       \%     Ampersand     \&
%%   Acute accent  \'     Left paren    \(     Right paren   \)
%%   Asterisk      \*     Plus          \+     Comma         \,
%%   Minus         \-     Point         \.     Solidus       \/
%%   Colon         \:     Semicolon     \;     Less than     \<
%%   Equals        \=     Greater than  \>     Question mark \?
%%   Commercial at \@     Left bracket  \[     Backslash     \\
%%   Right bracket \]     Circumflex    \^     Underscore    \_
%%   Grave accent  \`     Left brace    \{     Vertical bar  \|
%%   Right brace   \}     Tilde         \~}
%%
%
%
%
%<*begin12pt>

\expandafter\ifx\csname documentclass\endcsname\relax
    \documentstyle[12pt,dbtrace,german]{dinbrief}
    \typeout{Using the command \string\documentstyle.}
  \else
    \documentclass[12pt]{dinbrief}
    \usepackage{dbtrace}
    \usepackage{german}
    \usepackage[cp850]{inputenc}
    \typeout{Using the command \string\documentclass.}
  \fi

%</begin12pt>
%
%<*begin209>

    \documentstyle[12pt,dbtrace,german]{dinbrief}
    \typeout{Using the command \string\documentstyle.}

%</begin209>
%
%<*begin11pt>

\expandafter\ifx\csname documentclass\endcsname\relax
    \documentstyle[11pt,dbtrace,german]{dinbrief}
    \typeout{Using the command \string\documentstyle.}
  \else
    \documentclass[11pt]{dinbrief}
    \usepackage{dbtrace}
    \usepackage{german}
    \usepackage[cp850]{inputenc}
    \typeout{Using the command \string\documentclass.}
  \fi

%</begin11pt>
%
%<*begin10pt>

\expandafter\ifx\csname documentclass\endcsname\relax
    \documentstyle[dbtrace,german]{dinbrief}
    \typeout{Using the command \string\documentstyle.}
  \else
    \documentclass[10pt]{dinbrief}
    \usepackage{dbtrace}
    \usepackage{german}
    \usepackage[cp850]{inputenc}
    \typeout{Using the command \string\documentclass.}
  \fi

%</begin10pt>
%
%<*beginnorm>

\expandafter\ifx\csname documentclass\endcsname\relax
    \documentstyle[norm,dbtrace,german]{dinbrief}
    \typeout{Using the command \string\documentstyle.}
  \else
    \documentclass[norm]{dinbrief}
    \usepackage{dbtrace}
    \usepackage{german}
    \usepackage[cp850]{inputenc}
    \typeout{Using the command \string\documentclass.}
  \fi

%</beginnorm>
%
%<*brfpreamble>
\nofiles
\makelabels

\newlength{\UKAwd}
\newlength{\ADDRwd}
\font\fa=cmcsc10  scaled 1440
\font\fb=cmss12   scaled 1095
\font\fc=cmss10   scaled 1000
\def\briefkopf{
 \settowidth{\UKAwd}{\fa Institut f"ur Verpackungen}
 \settowidth{\ADDRwd}{\fc EARN/BITNET: yx99 at dkauni2}
 \expandafter\ifx\csname fontsize\endcsname\relax\else
   \fontsize{12}{14.4pt}\selectfont
 \fi
 \raisebox{-11.3mm}{%
   \setlength{\unitlength}{1truemm}
   \begin{picture}(15,15)(0,0)
     \thicklines
     \put(7.5,7.5){\circle{15}}
     \put(7.5,7.5){\circle{10}}
     \put(7.5,7.5){\circle{ 5}}
   \end{picture}%
 }
 \vspace*{7truemm}
 {\fc\hspace{.2em}}
 \parbox[t]{\UKAwd}{\centering{\fa Universit\"at Gralsruhe} \\
                    \centering{\fa Institut f"ur Verpackungen} \\[.5ex]
                    \centering{\fb Prof.\ Dr.\ Fritz Schreiber} }
 \hfill
 \parbox[t]{\ADDRwd}{\fc Im Hinterhof 2 $\cdot$ Postfach 8960 \\
                     \fc D--76821 Gralsruhe \\
                     \fc Telefon: (0127) 806-0815 \\
                     \fc Electronic Mail: \\
                     \fc EARN/BITNET: yx99 at error2 }
 }
\signature{Prof.\ Dr.\ Fritz Schreiber}
\place{Gralsruhe}
\date{den \today}

\address{\briefkopf}
\phone{(0127)}{806-0815}
\def\FS{Prof.\,F.\,Schreiber, Uni.\,Gralsruhe,
        Postf.\,8960, 76821\,Gralsruhe\rule[-1ex]{0pt}{0pt}}
\backaddress{\FS}

\pagestyle{contheadings}

\begin{document}

\bottomtext{%
  \makebox[\textwidth][c]{\small\sf
   Bankverbindung $\cdot$ Kreissparkasse Gralsruhe %
   (BLZ~999~500~00) 98~765~4
  }
}

%</brfpreamble>
%
%<*brief1>

\begin{letter}{%
  Herrn\\
  !Richard Gussmann\\
  Wartbergstra"se 25\\\medskip
  {\bf 74076 Heilbronn}}
\centeraddress
\date{\ntoday}
\concern{single letter $\cdot$ {\tt\string\centeraddress} is given\hfil\break
         {\tt\string\pagestyle\{contheadings\}} is given $\cdot$ testing
         {\tt\string\ntoday}\hfil\break
         {\tt\string\encl} and {\tt\string\cc} after {\tt\string\closing}}
\opening{Sehr geehrte Damen und Herren,}

{\bf Testing \verb|verse|}
\begin{verse}
{\bf Die F"u"se im Feuer\/}

Wild zuckt der Blitz,\\
im fahlen Lichte steht ein Turn,\\
der Donner rollt,\\
ein Reiter k"ampft mit seinem Ro"s,\\
springt ab un pocht ans Tor und l"armt.\\
Sein Mantel saust im Wind,\\
und knarrent "offnet jetzt das Tor ein Edelmann.\\
\dots\\
Der Reiter tritt in einen dunklen Ahnensaal.

Von eines weiten Herdes Feuer schwach erhellt,\\
droht hier ein Hugenott im Harnisch,\\
dort ein Weib, ein stolzes Weib in braunen Ebenbild.\\
Der Reiter wirft sich in den Sessel vor dem Herd\\
und starrt in den lebendgen Brand\\
\dots \\
Die Flamme zischt, zwei F"u"se zucken in der Glut.

\dots
\end{verse}

{\bf Testing \verb|quotation|}

\begin{quotation}
``Ich finde'', sagte einst Winston Churchill im
Unterhaus, ``die Art von Kritik, wie ich sie am
Sonntagmorgen bei meiner Ankunft in den Zeitungen
fand, erinnert mich immer an die Geschichte von
dem Matrosen, der in ein Hafenbecken sprang ---
in Plymouth, glaube ich ---, um einen kleinen
Jungen vom Ertrinken zu retten.

Dort sprach eine Frau den Matrosen an:\\
`Sind Sie der Mann, der meinen Sohn neulich
aus dem Wasser gezogen hat?'\\
Bescheiden erwiderte der Matrose:\\
`Ja, das stimmt.'\\
`Aha', sagte die Frau: `Ich suche Sie schon
die ganze Zeit \dots\ Wo ist seine M"utze?'{}''
\end{quotation}

{\bf Testing \verb|quote|}

\begin{quote}
Ein {\em klassisches\/} Werk ist ein Buch,\\
das die Leute loben,\\
aber nie lesen. \hfill({\em E.\ Hemingway\/})
\end{quote}

{\bf Filling up this side}

\begin{quotation}
\noindent{\bf Gl"uck und Ungl"uck}\smallskip

Mitten w"ahrend der Potsdamer Konferenz im
Sommer 1945 wurde Winston Churchill durch
den Entscheid der W"ahler als Premierminister
abberufen.

\flqq Wer wei"s\frqq, tr"ostete ihn seine
Gattin Clementine,\\
\flqq vielleicht ist es ein verkapptes Gl"uck.\frqq\\
\flqq Wenn es ein verkapptes Gl"uck ist\frqq,
antwortete Churchill, \flqq dann mu"s ich sagen,
da"s es verdammt gut verkappt ist!\frqq
\end{quotation}

\begin{quotation}
\noindent{\bf Wahrheit und L"uge}\smallskip

In besonders guter Laune f"uhrte Mark Twain
bei einem exklusiven Abendessen die
Gemahlin des Gouverneurs zu Tisch. Galant
sagte er: "`Wie sch"on Sie sind, Madame!"'\\
Geschmeichelt nahm die Dame das
Kompliment auf, entgegnete aber trotzdem
boshaft: "`Wie schade, da"s ich von ihnen nicht
dasselbe sagen kann."'\\
Mark Twain darauf: "`Machen Sie es doch wie
ich, gn"adige Frau: l"ugen Sie!"'
\end{quotation}

\begin{quotation}
\noindent{\bf Staatsdiener}\smallskip

Churchill hatte der Regierung vorgeworfen,
sie bestehe nur aus Politikern und habe keine
Staatsm"anner aufzuweisen.\\
"`Was ist der Unterschied?"' fragte man ihn.\\
"`Ein Politiker"', antwortete Churchill, "`denkt
immer nur an die n"achste Wahl, ein Staatsmann
denkt an die n"achste Generation."'
\end{quotation}

\begin{quotation}
\noindent{\bf Essen und Trinken}\smallskip

Sir Stafford Cripps, einer von Churchills
langj"ahrigen politischen Gegnern, war als
Vegetarier bekannt. W"ahrend der
Fleischknappheit der Nachkriegsjahre
bemerkte Churchill:\\
"`Jedermann kennt die hervorragenden
Talente, die Sir Stafford Cripps
uneingeschr"ankt seinen Landsleuten dienstbar
macht. Niemand hat gr"o"sere Anstrengungen
auf sich genommen als er, um den Kochtopf
der Nation zu f"ullen, und kaum einer nimmt
weniger heraus als er.\\
Ich habe auch meine Fraktionsvegetarier:
meinen Ehrenwerten Freund Lord Cherwell.
Diese "atherischen Wesen verf"ugen "uber eine
qualitativ wie quantitativ be\-acht\-li\-che
Gedankenproduktionskapazit"at bei
minimalen Treibstoff- und Wartungskosten."'
\end{quotation}

\begin{quotation}
\noindent{\bf Essen und Trinken}\smallskip

Wenn Churchill nicht schrieb, arbeitete er im
Garten seines Landhauses in Chartwell und
und besch"aftigte sich mit der Aufzucht von
Federvieh. Als an einem Sonntag eine
gebratene Gans serviert wurde, griff er zum
Tranchierbesteck, z"ogerte aber, legte das
nieder und sagte zu seiner Frau:\\
"`Schneid du sie
an, Clementine. Sie war eine Freundin
von mir."'
\end{quotation}

\cc[Durch Handzettel verteilt]{800 Copien}
\encl[]{Anlagen}

\closing{Mit freundlichen Gr"u"sen}

\ps{Bitte wo gehts hier zum WC? F"ur manche ist das dringend n"otig!}

\end{letter}

%</brief1>
%
%<*brief2>

\begin{letter}{%
  Herrn\\
  !Richard Gussmann\\
  Langestra"se 37\\[\medskipamount]
  {\bf 76199 Karlsruhe}}
\place{Karlsruhe}
\date{\today}
\concern{Einfacher Brief $\cdot$ place and date are redefined\hfil\break
         standard pagestyle $\cdot$ date and place redefined}
\opening{Sehr geehrte Damen und Herren,}

\begin{enumerate}
  \item {\bf Testing font switches\/}
        \begin{itemize}
          \item This is {\bf bold\/}.
          \item This is {\em emphasized and {\em double emphasized\/}}.
          \item This is {\sl slanted\/}.
          \item This is {\it italic\/}.
          \item This is {\sc caps and small caps\/}.
          \item This is {\sf sans serif\/}.
          \item This is {\tt typewriter\/}.
        \end{itemize}
  \item {\bf Testing math font switches\/}
        \begin{itemize}
          \item This is $\cal A $ (A is in {\tt\string\cal\/}).
          \item This is $\Gamma = {\mit \Gamma}$ (Second {\tt\string\Gamma} is
                {\it italic}).
        \end{itemize}
\end{enumerate}

\enclright

\encl{This is a enclosure}

\cc{This is a carbon copy}

\closing{Mit freundlichen Gr"u"sen}

\ps{This is a postscriptum}

\end{letter}

%</brief2>
%
%<*brief3>
\pagestyle{headings}

\begin{letter}{%
  Herrn\\
  !Richard Gussmann\\
  Wartbergstra"se 25\\\medskip
  {\bf 74076 Heilbronn}}
\centeraddress
\date{\ntoday}
\concern{single letter $\cdot$ {\tt\string\centeraddress} is given\hfil\break
         {\tt\string\pagestyle\{headings\}} is given $\cdot$ testing
         {\tt\string\ntoday}\hfil\break
         {\tt\string\encl} and {\tt\string\cc} after {\tt\string\closing}}
\opening{Sehr geehrte Damen und Herren,}

{\bf Testing \verb|verse|}
\begin{verse}
{\bf Die F"u"se im Feuer\/}

Wild zuckt der Blitz,\\
im fahlen Lichte steht ein Turn,\\
der Donner rollt,\\
ein Reiter k"ampft mit seinem Ro"s,\\
springt ab un pocht ans Tor und l"armt.\\
Sein Mantel saust im Wind,\\
und knarrent "offnet jetzt das Tor ein Edelmann.\\
\dots\\
Der Reiter tritt in einen dunklen Ahnensaal.

Von eines weiten Herdes Feuer schwach erhellt,\\
droht hier ein Hugenott im Harnisch,\\
dort ein Weib, ein stolzes Weib in braunen Ebenbild.\\
Der Reiter wirft sich in den Sessel vor dem Herd\\
und starrt in den lebendgen Brand\\
\dots \\
Die Flamme zischt, zwei F"u"se zucken in der Glut.

\dots
\end{verse}

\closing{Mit freundlichen Gr"u"sen}
\ps{Bitte wo gehts hier zum WC?}
\encl[]{Anlagen}
\cc[Durch Handzettel verteilt]{800 Copien}
\end{letter}

%</brief3>
%
%<*finale>

\end{document}

%</finale>
%
% \section{The \texttt{readme}-file}
%
%<*readme>

This file is part of the dinbrief package.
------------------------------------------

TABLE OF CONTENTS
=================

1. GENERAL INFORMATION
2. ERROR REPORTS
3. FILES IN THIS DISTRIBUTION
4. INSTALLATION


1. GENERAL INFORMATION
======================

IMPORTANT NOTICE:

You are not allowed to change this readme file.

Distribution of unchanged versions:


 Copyright (C) 1993, 96, 97 by University of Karlsruhe (Computing Center).
 Copyright (C) 1998, 2000   by University of Karlsruhe (Computing Center)
                            and Richard Gussmann.
 All rights reserved.
 For additional copyright information see further down in this file.

 This file is part of the DINBRIEF package


 It may be distributed under the terms of the LaTeX Project Public
 License (LPPL), as described in lppl.txt in the base LaTeX distribution.
 Either version 1.1 or, at your option, any later version.

 The latest version of this license is in

       http://www.latex-project.org/lppl.txt

 LPPL Version 1.1 or later is part of all distributions of LaTeX
 version 1999/06/01 or later.


 For error reports in case of UNCHANGED versions see readme files.

 Please do not request updates from us directly.  Distribution is
 done through Mail-Servers, TeX organizations and others.

 If you receive only some of these files from someone, complain!


  Redistribution of unchanged files is allowed provided that all files
  listed in the corresponding package README file are distributed
  including this readme file.

  However, if these files are distributed by established suppliers as
  part of a complete TeX distribution, and the structure of the
  distribution would make it difficult to distribute the whole set of
  files, *those parties* are allowed to distribute only some of the
  files provided that it is made clear that the user will get a
  complete distribution-set upon request to that supplier (not me).

  Notice that this permission is not granted to the end user.


Generation and distribution of changed versions:

  The generation of changed versions of the files included in the
  packages is allowed under the following restrictions:

  - You rename the file before you make any changes to it.

  - You acknowledge the origin of the original version in the file and
    keep the information that it (or a changed version) has to be
    distributed under the restrictions mentioned in this file.

  - You change the ERROR REPORT address so that we don't get error
    reports for files *not* maintained by us.


  The distribution of changed versions of the files included in the
  packages is allowed under the following restrictions:

  - You provide the user with information how to obtain the original
    package or, even better, distribute it with your files.

  - You make sure that the changed versions contain a notice that
    prevents others to take money for distribution or use of your
    files, i.e. they have to be distributed under the restrictions
    mentioned in this file.

  - You inform us that you created a changed version of the files.
    This is only necessary if you want to distribute it to others.


2. ERROR REPORTS
================

  Before you report an error please check that

  - the error is not already mentioned in the *.bug file of the
    distribution. (In this case it is a feature :-)

  - the error isn't caused by obsolete versions of other software;
    LaTeX from 1986 is a good candidate ...

  - you use an original version of the package.


  If you think you found a genuine bug please report it together
  with the following information:

  - version of the file

  - version (date!) of your LaTeX

  - a short test file showing the behavior with all unnecessary
    code removed.

  - a transcript (log file) of the session that shows the error.

Please note that it is important to make the file as small as possible
to allow us to find and fix the error soon.


Error reports in case of UNCHANGED versions to

                          K.D. Braune
                          Universit\"at Karlsruhe
                          Rechenzentrum
                          Postfach 6980
                          76128 Karlsruhe
               Internet:  <braune@rz.uni-karlsruhe.de>

                          Richard Gussmann
                          Max-Beckmann-Stra\ss{}e 31
                          76227 Karlsruhe
               Internet:  <richard@gussmann.de>

Please send error reports for contributed files to the original authors.


3. FILES IN THIS DISTRIBUTION
=============================

You should get the following files:

  dinbrief.dtx     `dinbrief' class/style for LaTeX in docstrip format.

  readme           This File

  dinbrief.ins     This is the installation script that will produce
                   the executable files in this package and the driver
                   files for the documentation when run through LaTeX
                   or TeX.


4. INSTALLATION
===============

To produce the executable files please run dinbrief.ins through LaTeX or
TeX, i.e., say

   latex dinbrief.ins

or whatever is necessary to process a file with LaTeX on your
system.  This will generate all necessary files. If you already have
older versions of the files, the script will ask whether or not you
want to overwrite those versions. Note, that the script calls
docstrip.tex internally which is distributed with LaTeX.

This script will produce the following files:

  dinbrief.cls  The dinbrief class file for LaTeX2e.
  dinbrief.sty  The dinbrief style file for LaTeX 2.09.
  dinbrief.drv  The driver file for producing the Users Guide
                and the documentation.
  dinbrief.tex  The Users Guide.
  dintab.tex    A list of available commands.

  example.tex   An example of a letter (needs brfkopf.tex).
  brfkopf.tex   An example letter head.
  brfbody.tex   LaTeX input file for the testXXXX.tex files.
  test10.tex    Files for testing the class/style.
  test11.tex    "
  test12.tex    "
  testnorm.tex  "
  dbold.tex     letter with R. Sengerlings commands.


To produce the documentation run the corresponding driver files
through LaTeX.  You are allowed to change the driver files to get a
special layout, etc.

%</readme>
%
% \section{Tracing the dinbrief}
%  Unfortunately \LaTeXe{} switches off all tracing posibilities.
%  To check the class carefully this little style can be used
%  to switch the tracing on again.
%
%    \begin{macrocode}
%<*dbtrace>
%    \end{macrocode}
%
%    \begin{macrocode}
 % \tracingcommands\tw@
 % \tracingstats\tw@
 % \tracingpages\@ne
 % \tracingoutput\@ne
 % \tracinglostchars\@ne
 % \tracingmacros\tw@
 % \tracingparagraphs\@ne
 % \tracingrestores\@ne
 % \showboxbreadth\maxdimen
\showboxbreadth15
 % \showboxdepth\maxdimen
\showboxdepth15
\errorstopmode
\errorcontextlines\maxdimen
 % \tracingonline\@ne
%    \end{macrocode}
%
%    \begin{macrocode}
%</dbtrace>
%    \end{macrocode}
%
% \section{The upload message}
%
%    \begin{macrocode}
%<*upload>
%    \end{macrocode}
%
%    \begin{macrocode}

Hallo,

ich habe soeben in das CTAN incomming Verzeichnis
eine aktualisierte Version der dinbrief Klasse eingespielt.

Sie bedindet sich bald in dem Verzeichnis
  /pub/tex/macros/latex/contrib/supported/dinbrief
Sie traegt die Versionsnummer 1.70 und ist ab der
LaTeX2e-Version vom 01.12.94 lauffaehig oder eben unter
LaTeX209.

Anmerkungen und Fehlerreports bitte an
    Richard Gussmann
    EMAIL: Internet: richard@gussmann.de
  oder an
    Dr. Klaus Braune
    EMAIL: Internet: braune@rz.uni-karlsruhe.de

1) Aenderungen
   1.1 Im AUX-File koennen die enthaltenen Befehle nun auch das
       at-Zeichen (Klammeraffe) enthalten.

2) Weitere Hinweise
   2.1 Neben den bisherigen Befehlen werden nun auch fast alle
       Befehle aus Rainer Sengerlings dinbrief unterstuetzt.
   2.2 Die Befehle \Anlagen und \Verteiler verhalten sich jetzt
       genau so wie in R. Sengerlings dinbrief.
   2.3 Die Befehle \encl und \cc haben ein optionales Argument
       bekommen. Damit kann die Default-Ueberschrift fuer einzelne
       Briefe geaendert werden. Ist das optionale Argument leer,
       so unterbleibt die Ausgabe der Ueberschriftenzeile.
       Form: \encl[andere Ueberschrift]{Anlagentexte}
             \cc[andere Ueberschrift]{Verteilvermerke}
       Beispiel: \encl[Anlagen]{Zeugnis\\Lebenslauf\\Lichbild}
                 \encl[]{Anlagen}
                 \cc[]{800 verteilte Handzettel}

3) Installation
   3.1 Kopiere die Dateien dinbrief.ins und dinbrief.dtx in ein beliebiges
       Verzeichnis.
   3.2 tex -i dinbrief.ins (DOS/OS/2 emTeX) oder
       initex dinbrief.ins oder
       latex  dinbrief.ins
       (Die notwendigen Umgebungsvariablen muessen natuerlich gesetzt sein.)
   3.3 Kopiere die Dateien dinbrief.cls und dinbrief.sty in ein
       Verzeichnis, das die TeX und LaTeX macros enthaelt
       z.B. c:\emtex\texinput oder
            c:\emtex\texinput\latex2e  (fuer den betatest).
   3.4 dinbrief.tex ist die Anleitung zum dinbrief.
       dinbrief.drv ist die Treiberdatei fuer Dokumentation und Anleitung.
       example.tex  ist ein Beispiel.
       brfkopf.tex  ist ein Beispiel fuer einen Briefkopf
                    (wird von example.tex benoetigt)
       dbold.tex    ist Rainer Sengerlings Beispiel zu seinen
                    dinbrief-Befehlen.

4) Abweichungen zu Rainer Sengerlings dinbrief
   4.1 Der Stil `german' muss mit einem
         \usepackage{german}
       geladen werden. Die Angabe \documentclass[12pt,german]{dinbrief}
       fuehrt NICHT zur gewuenschten Ausgabe.
   4.2 Der Befehl \heute wird nicht unterstuetzt.
   4.3 Die Befehle \Retourlabel und \Fenster erzeugen nur eine Warnung
       und aendern das Layout des Briefes NICHT bzw. erzeugen kein
       weiteres Label. (Dies kann sich in einer der naechsten
       Versionen aendern.)

5) Bekannte Fehler
   5.1) Der Stil `longtable' funktioniert nicht.
   5.2) Der Pagestyle contheadings arbeitet nicht in allen Faellen korrekt.
   5.3) Der Stil `merge' funktioniert nicht.

%    \end{macrocode}
%
%    \begin{macrocode}
%</upload>
%    \end{macrocode}
%
%    \begin{macrocode}
%<*oldchanges>
%    \end{macrocode}
%
%    \begin{macrocode}
1) Aenderungen (1.57)
   1.1 Der Betreff wird jetzt durch den Befehl \subject
       vereinbart. \concern steht weiterhin zur Verfuegung.
   1.2 Die Leerseite am Ende des Dokuments tritt nicht mehr auf.
   1.3 Die faelschlicherweise generierte Leerspalte bei den
       Labels wird nun mehr korrekt genutzt.

%    \end{macrocode}
%
%    \begin{macrocode}
%</oldchanges>
%    \end{macrocode}
%
% \section{The documentation driver file}
%
%    We have our own document class to format the \LaTeXe
%    documentation.
%    \begin{macrocode}
%<*driver>
\documentclass{ltxdoc}
\usepackage{german}
\originalTeX
%    \end{macrocode}
%
%    We don't want everything to appear in the index
%
%    We start with a nearly empty list and go then further and
%    further. So we can catch all interesting macros.
%
%    \begin{macrocode}
\DoNotIndex{\@empty,\\,\space,\@warning}
\DoNotIndex{\begin,\bfseries,\bgroup,\box,\def,\edef,\egroup}
\DoNotIndex{\else,\end,\endcsname,\expandafter,\fi,\gdef}
\DoNotIndex{\hbox}
\DoNotIndex{\hfil,\hfill,\hss,\ifx,\item,\let,\long,\message}
\DoNotIndex{\nointerlineskip}
\DoNotIndex{\originalTeX,\p@,\par,\parbox,\parsep,\relax}
\DoNotIndex{\setlength}
\DoNotIndex{\space,\string,\strut,\tmpa,\typeout,\verb}
\DoNotIndex{\vbox,\vskip,\vspace,\vss}
\DoNotIndex{\xdef,\z@,\z@skip}
\DoNotIndex{\LaTeX,\LaTeXe}
\DoNotIndex{\OnlyDescription,\PrintChanges,\PrintIndex,\RecordChanges}
%    \end{macrocode}
%    We do want an index, using linenumbers
%    \begin{macrocode}
\EnableCrossrefs
\CodelineIndex
 %\DisableCrossrefs
\RecordChanges
 %\OnlyDescription
\typeout{Expect some under- and overfull boxes}
%    \end{macrocode}
%    We also want the full details.
%    \begin{macrocode}
\begin{document}
\DocInput{dinbrief.dtx}
\PrintChanges
\PrintIndex
\end{document}
%</driver>
%    \end{macrocode}
%
\endinput
